\documentclass[12pt]{report}

\usepackage[utf8]{inputenc}
\usepackage{t1enc}
\usepackage[magyar]{babel}

\title{Structure from Motion from Two Views}
\date{}

\begin{document}
    \maketitle

    \chapter{Algoritmus}
    A Structure from motion (SfM) folyamat segítségével 3D rekonstrukciót hajthatunk végre egy képpár segítségével.

    \begin{enumerate}
        \item Két kép közötti ritka ponthalmazok megfeleltetése (pontmegfeleltetés): az első kép sarkainak azonosítása a \textit{detectMinEigenFeatures} függvénnyel, majd azok követése a második képre a \textit{vision.PointTracker} segítségével.
        \item Az esszenciális mátrix becslése \textit{estimateEssentialMatrix} használatával.
        \item Kamera elmozdulásának kiszámítása \textit{estrelpose} függvénnyel.
        \item Két kép közötti sűrű ponthalmazok megfeleltetése (pontmegfeleltetés): több pont kinyeréséhez újra kell detektálni a pontokat a \textit{detectMinEigenFeatures} függvény segítségével a \textit{'MinQuality'} opciót használva. Ezt követi a sűrű ponthalmaz követése a második képre a \textit{vision.PointTracker} használatával.
        \item Az illeszkedő pontok 3D helyzeteinek meghatározása a \textit{triangulate} segítségével (háromszögelés).
    \end{enumerate}    

    \chapter{Kód magyarázata}
        \section{Képpár betöltése}
            \begin{enumerate}
                \item \textit{fullfile(string1, string2, ...)} = az argumentumként kapott stringekből összeállít egy elérési útvonalat, pl.:\\\\
                    \texttt{path = fullfile('myfolder', 'mysubfolder')\\path = 'myfolder\textbackslash mysubfolder\textbackslash '}\\\\
                    \textit{toolboxdir(toolbox)} = visszaadja az argumentumként kapott toolbox abszolút elérési útvonalát.
                \item \textit{imageDatastore(path)} = létrehoz egy ImageDatastore objektumot a kapott elérési útvonallal meghatározott képekből. Az ImageDatastore objektum segítségével egy mappában található összes képet össze lehet gyűjteni egy változóba (de alapból nem lesz az összes kép egyszerre betöltve).
                \item \textit{readimage(datastore, n)} = betölti az n. képet a megadott datastore-ból.
                \item \textit{figure} = létrehoz egy új, üres ábra ablakot.
                \item \textit{imshowpair(image1, image2, 'montage')} = a meghatározott két képet egymás mellé helyezi a legutolsó ábrán.
                \item \textit{title('string')} = hozzáad egy címet a legutolsó ábrához.
            \end{enumerate}

        \section{A Camera Calibrator alkalmazás segítségével előre kiszámolt kamera paraméterek betöltése.}
            \begin{enumerate}
                    \item \textit{load(file\_name.mat)} = betölti egy korábban elmentett workspace adatait a jelenlegi workspace-be. A workspace egy ideiglenes tároló amely a MATLAB elindítása óta létrehozott változókat tárolja. Alapértelmezetten a MATLAB ablak jobb oldalán látható. A workspace-t el lehet menteni, így a benne tárolt változókat később vissza lehet tölteni a MATLAB-ba.
            \end{enumerate}

        \section{Lencse által okozott torzítás eltávolítása.}
            \begin{enumerate}
                \item \textit{undistortImage(image, intrinsics)} = a második argumentumként megadott kamera paramétereket felhasználva eltűnteti a kamera lencséje által okozott torzítást a megadott képről.\\
                      A kamera kalibrációja során kapott kamera paramétereket és a torzítási együtthatókat felhasználva kiszámítjuk a bemeneti kép minden pixelének eredeti pozícióját. Az egyes pixelek pozícióját az alábbi torzítások módosítják:
                        \begin{itemize}
                            \item \textbf{Radiális torzítás} = kiváltó oka, hogy a lencse szélén áthaladó fény jobban törik, mint a lencse közepén környezetében áthaladó fény. Ez kiszámolható:\\
                                \[x_d = x_u(1 + k_1r^2 + k_2r^4)\]
                                \[y_d = y_u(1 + k_1r^2 + k_2r^4)\]
                                (Ahol $x_u, y_u$ = torzulásmentes koordináták; $x_d, y_d$ = torzított koordináták; $k_1, k_2$ = radiális torzítási együtthatók; $r^2 = x_u^2 + y_u^2$)
                            \item \textbf{Tangenciális fordítás} = előfordul, ha a kameraszenzor és a lencse nem állnak tökéletesen párhuzamosan. Ez kiszámolható:\\
                                \[x_d = 2p_1x_uy_u + p_2(r^2+2x_u^2)\]
                                \[y_d = 2p_2x_uy_u + p_1(r^2 + 2y_u^2)\]
                                (Ahol $x_u, y_u$ = torzulásmentes koordináták; $x_d, y_d$ = torzított koordináták; $p_1, p_2$ = tangenciális torzítási együtthatók; $r^2 = x_u^2 + y_u^2$)\\\\
                    Az egyes pixelek korrigált helyének kiszámítása nem egész számú értékeket is előállít. Mivel a nem egész szám nem lehet pixel koordináta, ezért bilineáris interpolációt is végre kell hajtani. A bilineáris interpoláció során, a legközelebbi négy szomszédot felhasználva először lineáris interpolációt hajtunk végre az egyik irányba (pl. az x tengely mentén), majd pedig a másik irányba (az y tengely mentén):
                        \[out_P = I_1(1 - \Delta X)(1 - \Delta Y) + I_2 (\Delta X)(1 - \Delta Y) + I_3(1 - \Delta X)(\Delta Y) + I_4(\Delta X)(\Delta Y)\]
                        (Ahol $I_1, I_2, I_3, I_4$ = a szomszédos négy koordináta intenzitása az eredeti, torzított képen; $\Delta X, \Delta Y$ = a nem egész értékű koordinátákkal rendelkező vizsgált pixel és a vizsgált pixelhez legközelebb eső, egész értékű koordinátákkal rendelkező szomszédai közötti távolság; $out_P$ = végeredményként kapott pixel intenzitás)\\\\
                    A szomszédos pixelek efféle súlyzott átlagolásával, az interpoláció eredményeképp egy pixel intenzitás értéket kapunk, amely a legközelebbi egész érték koordinátával rendelkező pixel intenzitása lesz.\\
                    Az előállított, torzítatlan képen néhány pixel (leginkább a kép szélein) nem rendelkezik megfelelő pixel párral az eredeti, torzított képről (ezek azok a területek, ahol az eredeti képből nincs információ). Ezek a pixelek alapértelmezetten 0 értéket kapnak (feketék lesznek). 
                        \end{itemize}
            \end{enumerate}
    \chapter{Forrás}
        \begin{itemize}
            \item https://www.mathworks.com/help/vision/ug/structure-from-motion-from-two-views.html
            \item https://www.mathworks.com/help/visionhdl/ug/image-undistort.html
            \item https://e-learning.ujs.sk/pluginfile.php/23441/mod\textunderscore resource/content/1/01-ProjektivKamera.pdf
        \end{itemize}
\end{document}
