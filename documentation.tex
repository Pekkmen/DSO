\documentclass[12pt]{report}

\usepackage[utf8]{inputenc}
\usepackage{t1enc}
\usepackage[magyar]{babel}
\usepackage{amsmath}
\usepackage{url}

\title{Structure from Motion from Two Views}
\date{}

\begin{document}
    \maketitle

    \chapter{Algoritmus}
    A Structure from motion (SfM) folyamat segítségével 3D rekonstrukciót hajthatunk végre egy képpár segítségével.

    \begin{enumerate}
        \item Két kép közötti ritka ponthalmazok megfeleltetése (pontmegfeleltetés): az első kép sarkainak azonosítása a \textit{detectMinEigenFeatures} függvénnyel, majd azok követése a második képre a \textit{vision.PointTracker} segítségével.
        \item Az esszenciális mátrix becslése \textit{estimateEssentialMatrix} használatával.
        \item Kamera elmozdulásának kiszámítása \textit{estrelpose} függvénnyel.
        \item Két kép közötti sűrű ponthalmazok megfeleltetése (pontmegfeleltetés): több pont kinyeréséhez újra kell detektálni a pontokat a \textit{detectMinEigenFeatures} függvény segítségével a \textit{'MinQuality'} opciót használva. Ezt követi a sűrű ponthalmaz követése a második képre a \textit{vision.PointTracker} használatával.
        \item Az illeszkedő pontok 3D helyzeteinek meghatározása a \textit{triangulate} segítségével (háromszögelés).
    \end{enumerate}    

    \chapter{Kód magyarázata}
        \section{Képpár betöltése}
            \begin{enumerate}
                \item \textit{fullfile(string1, string2, ...)} = az argumentumként kapott stringekből összeállít egy elérési útvonalat, pl.:\\\\
                    \texttt{path = fullfile('myfolder', 'mysubfolder')\\path = 'myfolder\textbackslash mysubfolder\textbackslash '}\\\\
                    \textit{toolboxdir(toolbox)} = visszaadja az argumentumként kapott toolbox abszolút elérési útvonalát.
                \item \textit{imageDatastore(path)} = létrehoz egy ImageDatastore objektumot a kapott elérési útvonallal meghatározott képekből. Az ImageDatastore objektum segítségével egy mappában található összes képet össze lehet gyűjteni egy változóba (de alapból nem lesz az összes kép egyszerre betöltve).
                \item \textit{readimage(datastore, n)} = betölti az n. képet a megadott datastore-ból.
                \item \textit{figure} = létrehoz egy új, üres ábra ablakot.
                \item \textit{imshowpair(image1, image2, 'montage')} = a meghatározott két képet egymás mellé helyezi a legutolsó ábrán.
                \item \textit{title('string')} = hozzáad egy címet a legutolsó ábrához.
            \end{enumerate}

        \section{A Camera Calibrator alkalmazás segítségével előre kiszámolt kamera paraméterek betöltése.}
            \begin{enumerate}
                    \item \textit{load(file\_name.mat)} = betölti egy korábban elmentett workspace adatait a jelenlegi workspace-be. A workspace egy ideiglenes tároló amely a MATLAB elindítása óta létrehozott változókat tárolja. Alapértelmezetten a MATLAB ablak jobb oldalán látható. A workspace-t el lehet menteni, így a benne tárolt változókat később vissza lehet tölteni a MATLAB-ba.
            \end{enumerate}

        \section{Lencse által okozott torzítás eltávolítása.}
            \begin{enumerate}
                \item \textit{undistortImage(image, intrinsics)} = a második argumentumként megadott kamera paramétereket felhasználva eltűnteti a kamera lencséje által okozott torzítást a megadott képről.\\
                      A kamera kalibrációja során kapott kamera paramétereket és a torzítási együtthatókat felhasználva kiszámítjuk a bemeneti kép minden pixelének eredeti pozícióját. Az egyes pixelek pozícióját az alábbi torzítások módosítják:
                        \begin{itemize}
                            \item \textbf{Radiális torzítás} = kiváltó oka, hogy a lencse szélén áthaladó fény jobban törik, mint a lencse közepén környezetében áthaladó fény. Ez kiszámolható:\\
                                \[x_d = x_u(1 + k_1r^2 + k_2r^4)\]
                                \[y_d = y_u(1 + k_1r^2 + k_2r^4)\]
                                (Ahol $x_u, y_u$ = torzulásmentes koordináták; $x_d, y_d$ = torzított koordináták; $k_1, k_2$ = radiális torzítási együtthatók; $r^2 = x_u^2 + y_u^2$)
                            \item \textbf{Tangenciális fordítás} = előfordul, ha a kameraszenzor és a lencse nem állnak tökéletesen párhuzamosan. Ez kiszámolható:\\
                                \[x_d = 2p_1x_uy_u + p_2(r^2+2x_u^2)\]
                                \[y_d = 2p_2x_uy_u + p_1(r^2 + 2y_u^2)\]
                                (Ahol $x_u, y_u$ = torzulásmentes koordináták; $x_d, y_d$ = torzított koordináták; $p_1, p_2$ = tangenciális torzítási együtthatók; $r^2 = x_u^2 + y_u^2$)\\\\
                    Az egyes pixelek korrigált helyének kiszámítása nem egész számú értékeket is előállít. Mivel a nem egész szám nem lehet pixel koordináta, ezért bilineáris interpolációt is végre kell hajtani. A bilineáris interpoláció során, a legközelebbi négy szomszédot felhasználva először lineáris interpolációt hajtunk végre az egyik irányba (pl. az x tengely mentén), majd pedig a másik irányba (az y tengely mentén):
                        \[out_P = I_1(1 - \Delta X)(1 - \Delta Y) + I_2 (\Delta X)(1 - \Delta Y) + I_3(1 - \Delta X)(\Delta Y) + I_4(\Delta X)(\Delta Y)\]
                        (Ahol $I_1, I_2, I_3, I_4$ = a szomszédos négy koordináta intenzitása az eredeti, torzított képen; $\Delta X, \Delta Y$ = a nem egész értékű koordinátákkal rendelkező vizsgált pixel és a vizsgált pixelhez legközelebb eső, egész értékű koordinátákkal rendelkező szomszédai közötti távolság; $out_P$ = végeredményként kapott pixel intenzitás)\\\\
                    A szomszédos pixelek efféle súlyzott átlagolásával, az interpoláció eredményeképp egy pixel intenzitás értéket kapunk, amely a legközelebbi egész érték koordinátával rendelkező pixel intenzitása lesz.\\
                    Az előállított, torzítatlan képen néhány pixel (leginkább a kép szélein) nem rendelkezik megfelelő pixel párral az eredeti, torzított képről (ezek azok a területek, ahol az eredeti képből nincs információ). Ezek a pixelek alapértelmezetten 0 értéket kapnak (feketék lesznek). 
                        \end{itemize}
                \end{enumerate}

            \section{Pontmegfeleltetés a képek között.}
                \begin{enumerate}
                    \item \textit{detectMinEigenFeatures(grayImage, MinQuality=0.1)} = a Shi és Tomasi féle minimum sajátérték algoritmust (Shi \& Tomasi, Minimum Eigenvalue Algorithm) használva keresi meg a kép sarokpontjait (a sarokpont jelen esetben olyan pixeleket jelentenek, amelyek éles változást mutatnak a környező pixelekhez képest). Szürkeárnyalatos képet vár argumentumként, ezért a képet még előtte az \textit{im2gray} függvénnyel szürkeárnyalatosra változtatjuk. A \textit{MinQuality} argumentum a detektált sarokpontok minőségét határozza meg. Az értékének [0, 1] tartományból választhatunk. Magasabb érték, jobb minőségű, viszont kevesebb sarokpontot is eredményez.
                    Pontmegfeleltetés esetén gyakran a sarokpontok detektálása a preferált módszer, ugyanis a sarokpontok általában könnyen azonosíthatók különböző nézőpontból és stabilak (azaz kisebb elmozdulás, zaj vagy torzítás hatására is megismerhetők).\\
                    A Shi és Tomasi féle sajátérték algoritmus a Harris sarokpont detektáló algoritmuson alapszik. A Harris sarokpont detektáló algoritmus esetén első lépésben meghatározunk egy csúszóablakot a vizsgált képen. Ha azt tapasztaljuk, hogy ezt az ablakot bármelyik irányba is mozgassuk el, nagy lesz a különbség az ablak erdeti pozíciója alatti terület és az elmozdított ablak új pozíciója alatti terület között, akkor sikeresen detektáltunk egy sarokpontot. Ezt a változást az alábbi képlet szerint mérjük:
                    \[E(u,v) = \sum_{x,y}w(x,y)[I(x+u,y+v) - I(x,y)]^2\]
                    Ahol:\\ 
                    \begin{itemize}
                        \item \textit{E} = a négyzetkülönbség a csúszóablak eredeti és elmozdított pozíciója között.
                        \item \textit{u} = a csúszóablak elmozdításának értéke az x tengely mentén.
                        \item \textit{v} = a csúszóablak elmozdításának értéke az y tengely mentén.
                        \item \textit{w(x,y)} = a csúszóablak az (x,y) pozíción
                        \item \textit{I} = a kép intenzitása a zárójelekben meghatározott pozíción, pl:
                            \begin{itemize}
                                \item \textit{I(x,y)} = az ablak eredeti pozíciója alatti terület intenzitása.
                                \item \textit{I(x+u,y+v)} = az elmozgatott ablak alatti terület intenzitása.
                            \end{itemize}
                    \end{itemize}
                    A cél olyan csúszóablakokat találni, ahol ez az E érték nagy, bármelyik irányba is toljuk el az ablakot. Azaz olyan pozíció kell, ahol (a képletben) a szögletes zárójelben található kifejezés értéke nagy. Tehát a 
                    \[\sum_{x,y}[I(x+u,y+v) - I(x,y)]^2\]
                    részt kell maximalizálni. Ehhez először Taylor-sort alkalmazunk, amely után az alábbi egyenletet kapjuk (ezek után már csak közelítő eredményt kapunk):
                    \[E(u,v) \approx \sum_{x,y}[I(x,y) + ul_x + vl_y - I(x,y)]^2\]
                    A \textit{I(x,y) - I(x,y)} rész kiüti egymást, majd a elvégezzük a négyzetre emelést ($(a+b)^2 = a^2 + 2ab + b^2$):
                    \[E(u,v) \approx \sum_{x,y}u^2l_x^2 + 2uvl_xl_y + v^2l_y^2\]
                    Ezt mátrixá alakítjuk:
                    \[E(u,v) \approx \begin{bmatrix} u & v \end{bmatrix} (\sum \begin{bmatrix} l_x^2 & l_xl_y \\ l_xl_y & l_y^2 \end{bmatrix}) \begin{bmatrix} u \\ v \end{bmatrix}\]
                    A mátrixot (talán második momentum mátrixnak hívják és a kép intenzitásának változását méri az x és y irányokban) \textit{M} betűvel fogjuk jelezni és a \textit{w(x,y)} is csodával határos módon visszakerült:
                    \[M = \sum w(x,y)\begin{bmatrix} l_x^2 & l_xl_y \\ l_xl_y & l_y^2 \end{bmatrix}\]
                    Az \textit{M}-et behelyettesítve:
                    \[E(u,v) \approx \begin{bmatrix} u & v \end{bmatrix} M \begin{bmatrix} u \\ v \end{bmatrix}\]
                    Az így kapott \textit{E} értékből meg tudjuk mondani, hogy jelentős változás van-e a csúszóablak eredeti, valamint elmozdított pozíciója alatti terület között. Ha az intenzitásváltozás jelentős, akkor az alábbi képlettel döntjük el, hogy az adott ablak tartalmaz-e sarokpontot:
                    \[R = det(M) - k(trace(M))^2\]
                    Ahol:
                    \begin{itemize}
                        \item \textit{det(M)} = amely a második momentum mátrix determinánsát hazározza meg. Megadja, hogy mekkora az intenzitásváltozás mértéke mindkét irányban. Értéke: \textit{$det(M) = \lambda_1\lambda_2$}, ahol $\lambda_{1,2}$ az \textit{M} mátrix sajátértékei, és megmondják, hogy a képgradiens milyen irányba, milyen mértékben változik.
                        \item \textit{k} = egy állandó, amely a sarokpont detektálás szigorúságát határozza meg. Általában [0.04, 0.05] tartományba eső érték.
                        \item \textit{trace(M)} = a mátrix nyomát adja meg, amely a főátlók, azaz a két irány intenzitásváltozásának összege. Értéke: \textit{$trace(M) = \lambda_1 + \lambda_2$}
                    \end{itemize}
                    Tehát az \textit{M} mátrix \textit{$\lambda_1, \lambda_2$} sajátértékeinek értéke határozza meg, hogy a vizsgált régió sarokpont, él vagy felület-e:
                    \begin{itemize}
                        \item Ha R > 0, akkor az adott területen sarokpont van.
                        \item Ha R < 0, akkor az adott területen él van.
                        \item Ha R $\approx$ 0, akkor az adott terület homogén.
                    \end{itemize}

                    A Shi és Tomasi féle minimum sajátérték algoritmus egyetlen apró változtatást hajt végre a Harris féle sarokpont detektáló algoritmuson: A sarokpontok minőségét nem a sajátértékek kombinációjával, hanem a két sajátérték közül a kisebbik sajátérték alapján határozza meg. Az \textit{R} értékét ez az algoritmus az alábbi képlet szerint számolja ki:
                    \[R = min(\lambda_1, \lambda_2)\]
                    Ha \textit{R} nagyobb, mint egy előre meghatározott érték, akkor a vizsgált terület sarokpontot tartalmaz.
                    \item \textit{imshow(image, InitialMagnification = value)} = a meghatározott kép megjelenítése \textit{value}\%-os megnagyításban.
                    \item \textit{hold on} = a következőkben végrehajtott grafikus (pl.: grafikonok kirajzolása, stb.) parancsokat a legutolsó, aktuális ábrára fogja rárajzolni.
                    \item \textit{plot} = létrehoz kétdimenziós grafikont.
                    \item \textit{selectStrongest(featurePoints, N)} = visszaadja az \textit{N} legerősebb (talán ez az intenzitások különbségének nagyságát jelenti) jellemzőpontot (nálunk sarokpontok lesznek) a megadott \textit{featurePoints} változóból.
                    \item \textit{tracker = vision.PointTracker(MaxBidirectionalError=value1, NumPyramidLevels=value2)} = létrehoz egy Kanade-Lucas-Tomasi (KLT) algoritmus szerint működő pontkövető objektumot.\\
                    A KLT algoritmus kifejezetten jól működik olyan objektumok követésére, amely nem változtat alakot, valamint egyedi és részletes textúrával rendelkezik. A KLT algoritmus a Lucas-Kanade (LK) optikai áramlás becslés algoritmuson alapul. Az optikai áramlás az objetumok egy látszólagos/vizuális elmozdulása (nem feltétlenül egyezik meg a valós elmozdulással). Általános feltételezése, hogy egy objektum pixeleinek intenzitása elmozdulástól függetlenül állandó:
                    \[I(x,y,t) = I(x + u,y + v,t + 1)\]
                    Ahol:
                    \begin{itemize}
                        \item \textit{I(x,y,t)} = \textit{(x,y)} pozíción lévő pixel intenzitása \textit{t} időben
                        \item \textit{u} = elmozdulás az x tengelyen
                        \item \textit{v} = elmozdulás az y tengelyen
                    \end{itemize}

                    Ebből kifejezhető az optikai áramlás egyenlete Taylor sort alkalmazva:
                    \[I(x + u,y + v,t + 1) \approx I(x,y,t) + I_xu + I_yv + I_t\]
                    \[I(x + u,y + v,t + 1) - I(x,y,t) = I_xu + I_yv + I_t\]
                    \[0 \approx I_xu + I_yv + I_t\]
                    Ahol:
                    \begin{itemize}
                        \item \textit{$I_xu$} = az intenzitás változása az x tengelyen.
                        \item \textit{$I_yv$} = az intenzitás változása az y tengelyen.
                        \item \textit{$I_t$} = az intentizás idő szerinti változása.
                    \end{itemize}
                    Tehát, ha az \textit{u} és \textit{v} elmozdulások helyesen vannak meghatározva, akkor az intenzitások különbsége megközelítőleg 0. Azonban \textit{u} és \textit{v} ismeretlenek és meghatározásukat nehezíti a \textbf{nyílás/rekesz/apertúra probléma (aperture problem)}, azaz amikor a tényleges mozgást egy kicsi rekeszen keresztül figyelve próbáljuk meghatározni. Egy objektum tényleges elmozdulásának kiszámítása nehéz, ha csak egy kis területet látunk belőle a résen keresztül.\\
                    A Lucas-Kanade algoritmus feltételezi, hogy az optikai áramlás (\textit{u} és \textit{v}) konstans és a képen textúrázott tárgyak láthatók. A megfigyelt pixel kezdeti intenzitása \textit{a}, valamint a rés mögött látható tárgy elmozdítása után észlelt pixel intenzitása \textit{b}. Ezek különbsége \textit{(b - a)}, azaz az időbeli intenzitáskülönbség \textit{$I_t(x,y)$}. A \textit{(x,y)} pozíción megfigyelt pixel intenzitásváltozás \textit{$I_x(x,y)$} az x tengelyen, valamint \textit{$I_y(x,y)$} az y tengelyen. Az x tengelyen történő \textit{u} és y tengelyen történő \textit{v} elmozdulás esetén a pixel intenzitása:
                    \[I_x(x,y)u + I_y(x,y)v = -I_t(x,y)\]
                    Nem tudom miért negatív az $I_t$ :(\\\\
                    Az algoritmus helyes működéséhez nem elegendő egyetlen pixelt nézni, így ezt ki kell terjeszteni egy pixelszomszédságra. Egy 3x3 szomszédság esetén az egyenlet:
                    \[I_x(x + \Delta x, y + \Delta y)u + I_y(x + \Delta x, y + \Delta y)v = -I_t(x + \Delta x, y + \Delta y)\] ahol $\Delta x$ = -1,0,1 (a három szomszédos pixel az x tengelyen), $\Delta y$ = -1,0,1 (a három szomszédos pixel az y tengelyen)
                    Tömör formában a képlet:
                    \[S\begin{bmatrix} u \\ v \end{bmatrix} = \overset{\rightarrow}{t}\]
                    Ahol:
                    \begin{itemize}
                        \item \textit{S} = 9x2 mátrix amely tartalmazza: $[I_x(x + \Delta x, y + \Delta y),  I_y(x + \Delta x, y + \Delta y]$.
                        \item \textit{$\overset{\rightarrow}{t}$} = vektor amely tartalmazza: $-I_t(x + \Delta x, y + \Delta y)$.
                    \end{itemize}
                    Ezen két ismeretlenes egyenlet megoldása a legkisebb négyzetek módszerével történik, amihez megszorozzuk az egyenletet $S^{T}$ (én sem értem miért):
                    \[S^{T}S\begin{bmatrix}u \\ v\end{bmatrix} = S^{T}\overset{\rightarrow}{t}\]
                    Ebből $\begin{bmatrix}u \\ v\end{bmatrix}$ kifejezve:
                    \[\begin{bmatrix}u \\ v\end{bmatrix} = (S^{T}S)^{-1}S^{T}\overset{\rightarrow}{t}\]
                    Röviden (ha jól fogtam fel) a Lucas-Kanade optikai áramlás becslés algoritmusa a képkockák közötti mozgást próbálja nyomon követni. Ehhez két képkockán kiválaszt egy pixelszomszédságot és azok térbeli elmozdulását elemzi. Ezután kiszámolja a pixelértékek intenzitásváltozását, amelyből próbálja kiszámolni az objektum tényleges elmozdulását. Gyorsan mozgó vagy nem ritkásan textúrázott objektumok esetén nem megbízható.\\
                    A tényleges pontmegfeleltetést a Kanade-Lucas-Tomasi (KLT) algoritmus végzi, amelyből a MATLAB vision.PointTracker függvény a pyramid KLT továbbfejlesztett verziót használja. Először minden kiválasztott jellemzőponthoz egy kis ablakot/apertúrát illeszt. A következő lépésben létrehoz egy kép piramist: a kiinduló képből több szintet hoz létre, ahol az alsó szinten található a teljes (eredeti) felbontású kép, a piramis teteje felé haladva a vizsgált kép egyre kisebb felbontású verziója fog szerepelni. Ezután kiszámolja a kiválasztott pixelek elmozdulás vektorait a piramis legfelső szintjétől kezdve, az előbb tárgyalt Lucas-Kanade optikai áramlás becslés algoritmusát felhasználva. A becsült mozgásokat használva finomítja a jellemzőpontok pozícióit a következő alacsonyabban elhelyezkedő, magasabb felbontású piramisszinten. Ezt addig folytatja, amíg el nem éri a legalsó szintet. A piramisos módszer előnye, hogy az alacsonyabb felbontású képeken végzett kezdeti nyomkövetés javítja a teljes nyomkövetési pontosságot és akár nagyobb távolságok követését is lehetővé teszi.\\
                    A \textit{vision.PointTracker} függvényhívásnál használt két argumentum:
                    \begin{itemize}
                        \item \textit{MaxBidirectionalError} = a funkció lényege, hogy az épp vizsgált pontot az első képről követi a második képre, majd ugyanazon pontot követi visszafelé a második képről az első képre. Ezután kiszámolja a vizsgált pont eredeti és a visszakövetés utáni, új pozíciója közötti különbséget (az első képen). Ha ez az érték nagyobb, mint az függvénynek átadott \textit{MaxBidirectionalError} érték, akkor az adott pont érvénytelen. Hatékony módszer a nem megbízhatóan követhető pontok kiküszöbölésére. Az ajánlott érték 0 és 3 pixel között van.
                        \item \textit{NumPyramidLevels} = a képpiramis szintjeinek számát határozza meg. A nagyobb érték lehetővé teszi a nagyobb elmozdulások követését, viszont lassabb futási időt eredményez. Az ajánlott érték 1 és 4 között van.
                    \end{itemize}
                    \item \textit{imagePoints1 = imagePoints1.Location} = a sarokpont objektumok pozícióinak kiolvasása.
                    \item \textit{initialize(pointTracker, points, image1)} = inicializálja az első argumentumként kapott pontkövető objektumot, inicializálja a második argumentumként kapott követendő pontokat, valamint beállítja az utolsó argumentumként kapott képet a kezdeti képkockának.
                    \item \textit{[returnValue1, returnValue2] = step(tracker, image2)} = a \textit{step} függvény indítja el az első argumentumként kapott pontkövető objektumot. A függvény második argumentum a második képkocka, amelyen el kell végeznie a pontkövetést az objektumnak. Az első visszatérési érték a korábban meghatározott jellemzőpontok koordinátái a második képkockán. A második visszatérési érték egy logikai indexvektor, amely azt jelzi, hogy mely pontokat sikerült nyomon követni. A megbízhatóan követhető pontok \textit{true}, míg ellenkező esetben \textit{false} értéket kapnak.
                    \item \textit{matchedPoints = imagePoints(validIdx, :)} = a megbízhatóan követhető pontok lementése. Az \textit{imagePoints} egy mátrix, melynek sorai tartalmazzák a jellemzőpontokat, két oszlopai pedig a jellemzőpontok x és y koordinátáit. A \textit{imagePoints(validIdx, :)} rész egy mátrix címzés, amely visszaadja mátrix azon sorait ahol \textit{validIdx} = \textit{1} vagy \textit{true}, valamint a mátrix összes oszlopát (: jelentése összes index).
                    \item \textit{showMatchedFeatures(image1, image2, matchedPoints1, matchedPoints2)} = a pontmegfeleltetés eredményének kirajzolása. A két agumentumként kapott kép színkódolást kap, valamint a másik két, argumentumként kapott jellemzőpontok is kirajzolódnak és egy vonallal lesznek összekötve a hozzájuk tartozó párjukkal.
                    \item \textit{[returnValue1, returnValue2] = estimateEssentialMatrix(matchedPoints1, matchedPoints2, intrinsics)} = az esszenciális mátrix becslése egy képpár megfeleltetett pontjaiból.
                    \\To be continued...\\
                    \url{https://www.baeldung.com/cs/fundamental-matrix-vs-essential-matrix}
                \end{enumerate}

    \chapter{Forrás}
        \begin{itemize}
            \item \url{https://www.mathworks.com/help/vision/ug/structure-from-motion-from-two-}\\\url{views.html}
            \item \url{https://www.mathworks.com/help/vision/ref/undistortimage.html?s_tid=doc_ta}
            \item \url{https://www.mathworks.com/help/visionhdl/ug/image-undistort.html}
            \item \url{https://e-learning.ujs.sk/pluginfile.php/23441/mod_resource/content/1/01-ProjektivKamera.pdf}
            \item \url{https://www.mathworks.com/help/vision/ref/detectmineigenfeatures.html}
            \item \url{https://aishack.in/tutorials/features/}
            \item \url{https://aishack.in/tutorials/harris-corner-detector/}
            \item \url{https://aishack.in/tutorials/shitomasi-corner-detector/}
            \item \url{https://docs.opencv.org/3.4/dc/d0d/tutorial_py_features_harris.html}
            \item \url{https://www.mathworks.com/help/vision/ref/vision.pointtracker-system-object.html}
            \item \url{https://lorenzopeppoloni.com/lkttracker/}
            \item \url{https://www.baeldung.com/cs/optical-flow-lucas-kanade-method}
            \item \url{https://www.inf.u-szeged.hu/~kato/teaching/IpariKepfeldolgozas/08-Motion.pdf}
            \item \url{http://www.inf.fu-berlin.de/inst/ag-ki/rojas_home/documents/tutorials/Lucas-Kanade2.pdf}
            \item \url{https://link.springer.com/chapter/10.1007/978-3-319-29451-3_29}
            \item \url{https://www.mathworks.com/help/matlab/ref/step.html}
            \item \url{https://www.mathworks.com/help/vision/ref/showmatchedfeatures.html}
            \item \url{https://www.baeldung.com/cs/fundamental-matrix-vs-essential-matrix}
            \item \url{https://e-learning.ujs.sk/pluginfile.php/23450/mod_resource/content/1/05-SztereoKamera.pdf}
        \end{itemize}
\end{document}
