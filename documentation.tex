\documentclass[10pt]{report}

\usepackage[utf8]{inputenc}
\usepackage{t1enc}
\usepackage[magyar]{babel}

\title{Structure from Motion from Two Views}
\date{}

\begin{document}
    \maketitle

    A Structure from motion (SfM) folyamat segítségével 3D rekonstrukciót hajthatunk végre egy képpár segítségével. Ennek lépései:

    \begin{enumerate}
        \item Két kép közötti ritka ponthalmazok megfeleltetése (pontmegfeleltetés): az első kép sarkainak azonosítása a \textit{detectMinEigenFeatures} függvénnyel, majd azok követése a második képre a \textit{vision.PointTracker} segítségével.
        \item Az esszenciális mátrix becslése \textit{estimateEssentialMatrix} használatával.
        \item Kamera elmozdulásának kiszámítása \textit{estrelpose} függvénnyel.
        \item Két kép közötti sűrű ponthalmazok megfeleltetése (pontmegfeleltetés): több pont kinyeréséhez újra kell detektálni a pontokat a \textit{detectMinEigenFeatures} függvény segítségével a \textit{'MinQuality'} opciót használva. Ezt követi a sűrű ponthalmaz követése a második képre a \textit{vision.PointTracker} használatával.
        \item Az illeszkedő pontok 3D helyzeteinek meghatározása a \textit{triangulate} segítségével (háromszögelés).
    \end{enumerate}    
\end{document}
